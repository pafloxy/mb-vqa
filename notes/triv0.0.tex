% Options for packages loaded elsewhere
\PassOptionsToPackage{unicode}{hyperref}
\PassOptionsToPackage{hyphens}{url}
%
\documentclass[
]{article}
\usepackage{lmodern}
\usepackage{amssymb,amsmath}
\usepackage{ifxetex,ifluatex}
\ifnum 0\ifxetex 1\fi\ifluatex 1\fi=0 % if pdftex
  \usepackage[T1]{fontenc}
  \usepackage[utf8]{inputenc}
  \usepackage{textcomp} % provide euro and other symbols
\else % if luatex or xetex
  \usepackage{unicode-math}
  \defaultfontfeatures{Scale=MatchLowercase}
  \defaultfontfeatures[\rmfamily]{Ligatures=TeX,Scale=1}
\fi
% Use upquote if available, for straight quotes in verbatim environments
\IfFileExists{upquote.sty}{\usepackage{upquote}}{}
\IfFileExists{microtype.sty}{% use microtype if available
  \usepackage[]{microtype}
  \UseMicrotypeSet[protrusion]{basicmath} % disable protrusion for tt fonts
}{}
\makeatletter
\@ifundefined{KOMAClassName}{% if non-KOMA class
  \IfFileExists{parskip.sty}{%
    \usepackage{parskip}
  }{% else
    \setlength{\parindent}{0pt}
    \setlength{\parskip}{6pt plus 2pt minus 1pt}}
}{% if KOMA class
  \KOMAoptions{parskip=half}}
\makeatother
\usepackage{xcolor}
\IfFileExists{xurl.sty}{\usepackage{xurl}}{} % add URL line breaks if available
\IfFileExists{bookmark.sty}{\usepackage{bookmark}}{\usepackage{hyperref}}
\hypersetup{
  hidelinks,
  pdfcreator={LaTeX via pandoc}}
\urlstyle{same} % disable monospaced font for URLs
\setlength{\emergencystretch}{3em} % prevent overfull lines
\providecommand{\tightlist}{%
  \setlength{\itemsep}{0pt}\setlength{\parskip}{0pt}}
\setcounter{secnumdepth}{5}

\author{}
\date{}

\begin{document}

\%\% Add defnitions of stabilizers \%\%

\begin{quote}
{[}@rp{]}: Please refer to {[}stabg\#1{]} for a preliminary introduction
to stabilizer states, and/or also the material in {[}stabg\#2{]}. It
will be enough for you to understand the basics of stabilizer formalism,
i.e

\begin{itemize}
\tightlist
\item
  How to represent an arbitrary quantum state in the formalism,
\item
  Realize that any quantum arbitrary state can have a stabilized and a
  non-stabilized (/logical) subspace
\item
  How to see the analogue of an unitary transformation in the stabilizer
  formalism It won't be necessary to dive into concepts specific to
  error-correction such as code-distance, etc. (\emph{unless you find
  them interesting ;)}
\end{itemize}

Below I will summarise just a few basics notations and concepts that I
will be using in rest of the document.
\end{quote}

\textbf{Notation :} Given two pauli operators on n-qubits
\(A,B \in \mathcal{P}^{\otimes n}\) they can either commute or
anti-commute and we will use the notation \(\odot\) as
\[A \odot B = \begin{cases} 0 \: ; [A,B] =0 \\ 1 \: ; [A,B] \neq 0 \end{cases}\]to
indicate their inter-commutation relation.

\hypertarget{isomorphism-of-pauli-operators}{%
\paragraph{Isomorphism of Pauli
Operators}\label{isomorphism-of-pauli-operators}}

We know that for single qubits there are three n on-trivial pauli
operators namely \(\mathcal{P}^{\otimes 1} = \{I, X, Y ,Z\}\). However
as will be apparent in the later sections it is often convenient to use
the fact that \(Y = i ZX\), to say that \(\mathcal{P}^{\otimes1}\) is
generated by the elements \(X,Z\) upto an overall phase (i.e if we
ignore the \(i\)). This fact allows us to say that for a 1-qubit system
where the hilbert space is two-dimensional we have two ``independent''
pauli operators that are \(X,Z\), where the notion of independence
corresponds to the fact that the \(Z\) cannot be expressed as a
scalar-multiple of \(X\) or vice versa, whereas \(Y\) can be.

Similarly for a n-qubit system we can construct respective pauli
operators by taking n-tensor products of the single qubit pauli
operators as
\(\mathcal{P}^{\otimes n} = \bigotimes_{i=1}^{n}\mathcal{P}^{\otimes 1}\),
for example
\(\mathcal{P}^{\otimes2}=\{I, X_{0}, Z_{0},Y_{0}\}\otimes \{I, X_{1}, Z_{1}, Y_{1}\}\).
Generalizing the result before we can see that an \(n\) qubit system
must have \(2n\) many `independent' pauli operators, represented as
\(\{X_{0}, Z_{0}, X_{1}, Z_{1} \dots X_{n}, Z_{n}\}\). Observe that they
must satisfy the commutation relation
\[X_{i} \odot Z_{j}= \delta_{ij}\],which we will refer to as the
`canonical commutation relation' and is easy to deduce from the fact
that the pauli operators only interact non-trivially when they're acting
on the same qubit and thus must commute otherwise (remember
\(X_{i}= I_1 \otimes \dots \otimes X_{i} \otimes \dots I_n\)).

We can define a canonical transformation
\(T: \mathcal{P}^{\otimes n} \to \mathcal{P}^{\otimes n}\) which
transforms the independent pauli operators to independent
pauli-operators as
\[\begin{align} \forall_{i=1}^{n} \:\:\: X_{i} \to \tilde{X_{i}}\:; \: Z_{i}\to \tilde{Z_i}\end{align}\]
where \(\tilde{X_{i}} \otimes \tilde{Z_{j}} = \delta_{ij}\). Note that
\(\tilde{X_{i}}, \tilde{Z_{i}}\) are not necessarily single qubits
operators unlike \(X_{i}, Z_{i}\), but could be any element from
\(\mathcal{P}^{\otimes n}\) as long as they satisfy the commutation
relation. For example we can define the following transformation on
\(\mathcal{P}^{\otimes 2}\)
\[\begin{align} Z_{1} \to X_{1}X_{2} \:&;\: Z_{2}\to Z_{1}Z_{2} \\ X_{1} \to Z1 \:&; \: X_{2} \to X_1 \end{align}\],
where the the operators \(Z_{1}, Z_{2}\) are mapped to non-local
pauli-operators. All such pauli groups generated via
canonical-transformations are called ``isomorphic'' and such
transformations are called ``Clifford'' transformations, something we
will come back to later.

\hypertarget{stabilizer-states}{%
\paragraph{Stabilizer States}\label{stabilizer-states}}

For any \(n\)-qubit system, a state \(\ket{\psi}\) is called a
stabilizer state if we can find \(n\) `independent' and `commuting'
pauli operators
\(\mathcal{S} = \{S_{1}, S_{2}, \dots S_{n}\} \subset \mathcal{P}^{\otimes n}\)
such that
\(\forall_{S \in \braket{\mathcal{S}}} \: S \ket{\psi} = \ket{\psi}\)

For convenience we can use the notation
\(\ket{\mathcal{S}} = \ket{S_{1}S_{2} \dots S_n}\) to refer to a
stabilizer state stabilized by \(\braket{\mathcal{S}}\).

From the properties of pauli operators discussed above its easy to see
the for corresponding to any set of n independent and commuting pauli
operators \(\mathcal{S}\) we must have a complementary set of n
independent and commuting operators \(\mathcal{S}^c\) such that for
every element of \(\mathcal{S}\) there exists exactly one element in
\(\mathcal{S}^c\) that anti-commutes with it. This relation can be
encapsulated by recognizing the paulis in \(\mathcal{S}\) and
\(\mathcal{S}^{c}\) with \(\mathcal{\tilde{Z}}\) and
\(\mathcal{\tilde{X}}\) respectively, (i.e
\(S_{i}\to \tilde{Z}_{i}\:, \: S^{c}_{i} \to \tilde{X}_{i}\)) \#\#\#\#
Action of Pauli operators on Stabilizer States

Since
\(\forall_{i} \: \tilde{Z}_{i} \ket{\tilde{\mathcal{Z}}} = \ket{\tilde{\mathcal{Z}}}\)
\[\begin{align} \tilde{Z}_{i}  \ket{\tilde{Z}_{1}.. \tilde{Z}_{j} .. -\tilde{Z}_{i}\dots \tilde{Z}_{n}}  \: &= \: -\ket{\tilde{Z}_{1}.. \tilde{Z}_{j}..-\tilde{Z}_{i}\dots \tilde{Z}_{n}}  \\ \tilde{Z}_{j}  \ket{\tilde{Z}_{1}.. \tilde{Z}_{j} .. -\tilde{Z}_{i}\dots \tilde{Z}_{n}}  \: &= \: \ket{\tilde{Z}_{1}.. \tilde{Z}_{j}..-\tilde{Z}_{i}\dots \tilde{Z}_{n}} \end{align}\]
the only paulis that act non-trivially (i.e not equivalent upto overall
phase) on the stabilizer state \(\ket{\mathcal{\tilde{Z}}}\) must be
elements of the set \(\mathcal{\tilde{X}}\).

The pauli \(\tilde{X}_i\) anticommutes exclusively with
\(\tilde{Z}_{i}\), thus it should transform the state
\(\ket{\mathcal{\tilde{Z}}}\) in manner that measuring the
\(\tilde{Z}_i\) on this transformed state must yield \(-1\), however
measuring any of \(\forall_{j \neq i} \:\tilde{Z}_{j}\) should still
yield \(+1\), since they commute with \(\tilde{X}_i\).

This allows us to represent the transformation induced by
\(\tilde{X}_{i}\) as
\[\tilde{X}_{i} \ket{\tilde{Z}_{1}.. \tilde{Z}_{i}\dots \tilde{Z}_{n}}  \: = \: \ket{\tilde{Z}_{1}.. -\tilde{Z}_{i}\dots \tilde{Z}_{n}}\]
Similarly, transformation induced by any arbitrary operator of the form
\(\tilde{X} = \tilde{X}_{i}\tilde{X}_{j} ..\tilde{X}_k\) can be easily
obtained as
\(\tilde{X} \ket{\tilde{\mathcal{Z}}} = \ket{ \forall_{i} \: (-1)^{\tilde{X} \odot \tilde{Z}_i}\tilde{Z}_{i}}\)

Also note that any two stabilizer states on which differ by the action
of a non-trivial pauli must by orthogonal to each other, this can be
seen from the expectation values of
\(\braket{\tilde{\mathcal{Z}}| \tilde{X}|\tilde{\mathcal{Z}}}\) in
\[\begin{align} 2\braket{\tilde{\mathcal{Z}}| \tilde{X}|\tilde{\mathcal{Z}}} &= \braket{\tilde{\mathcal{Z}}| \tilde{X}|\tilde{\mathcal{Z}}} +\braket{\tilde{\mathcal{Z}}| \tilde{X}\tilde{Z}_i|\tilde{\mathcal{Z}}} \\ &= \braket{\tilde{\mathcal{Z}}| \tilde{X}|\tilde{\mathcal{Z}}} -\braket{\tilde{\mathcal{Z}}| \tilde{Z}_i\tilde{X}|\tilde{\mathcal{Z}}} \\ &= 0\end{align}\]where
\(\tilde{Z}_i\) is such that \(\tilde{Z}_{i}\odot \tilde{X}=1\). (that
there will exist a \(\tilde{Z}_i\) like this follows from the fact that
\(\tilde{X}\) is non-trivial) Since there are \(2^{n}\) such non-trivial
operators in \(\tilde{\mathcal{X}}\), we can generate \(2^n\) mutually
orthogonal stabilizer states by acting \(\ket{\mathcal{\tilde{Z}}}\)
with the operators in \(\tilde{\mathcal{X}}\) which will be of the form
\(\{\ket{\pm \tilde{Z}_{1} ..\pm \tilde{Z}_{n}}\}\). (\%\%Group
theoretically speaking, all the basis vectors are `cosets' of the
stabilizer group \(\braket{\mathcal{\tilde{Z}}}\) under the action of
the operators in \(\mathcal{\tilde{X}}\) \%\% ) Thus we can use this
`stabilizer-basis' to represent any arbitrary state in the n-qubits
hilbert space \(\mathcal{H}^{\otimes2^n}\).

Realize that the mostly commonly used computational basis state
\(\{ \ket{\bigotimes_{i=1}^{n} z_{i}} \:\: \forall_{i}z_{i} \in \{0,1\} \: \}\)
is also a `stabilizer-basis' where we have restricted all the stabilizer
operators \(\forall_{i} \tilde{Z}_{i}\) to be acting on the physical
\(i\)th qubit imposing \(\forall_{i} \tilde{Z}_{i} = Z_{i}\) to yield
the basis \(\{\ket{\pm Z_{1}.. \pm Z_n}\}\). For an example, the state
\(\ket{\psi} = \alpha \ket{0} + \beta \ket{1}\) can equivalently
represented as \(\ket{\psi} = \alpha \ket{+Z} + \beta \ket{-Z}\). A key
thing to note is that in general an arbitrary linear composition of
different stabilizer states is not a stabilizer state, in the previous
example the state \(\ket{\psi}\) is not stabilized by either \(\pm Z\)
for arbitrary choices of \(\alpha, \beta\).

\hypertarget{example-bell-state}{%
\paragraph{Example: Bell State}\label{example-bell-state}}

The bell state or more generally the bell basis is characterized by the
stabilizer generators
\(\mathcal{S}_{\beta_{00}}= \{X_{0}X_{1}, Z_{0}Z_{1} \}\), the
corresponding stabilizer-state is
\(\ket{\beta_{00}} = \ket{00} + \ket{11}\). It is easy to see that both
the stabilizers generators yield an eigenvalue \(+1\) on the state
\(\ket{\beta_{00}}\).

Question is how can we generate the entire basis for the two-qubit
hilbert space \(\mathcal{H}^{\otimes2^2 }\) ? We can try investigating
this by redefining a set of four stabilizer generators as
\[\mathcal{S}_{\beta_{ij}} = \{ -1^{i}X_{0}X_{1},-1^{j} Z_{0}Z_{1} \} \].
Where for any \(i,j\) the structure of the generators
\(\mathcal{S}_{\beta_{ij}}\) indicate the measurement outcome we should
get upon measuring each individual generator, for instance a state
\(\ket{\beta_{01}}\) stabilized by \(\mathcal{S}_{\beta_{01}}\) should
satisfy the conditions
\[X_0X_{1}\ket{\beta_{01}}= \ket{\beta_{01}} \:; \: Z_0Z_{1}\ket{\beta_{01}}= -\ket{\beta_{01}}\].
But how do we guess that state \(\ket{\beta_{01}}\) (or any
\(\ket{\beta_{ij}}\))? Remember from the previous example that if we
define \(\tilde{Z_{0}}=X_0X_{1} , \tilde{Z_{1}}= Z_{0}Z_{1}\) we can
have \(\tilde{X_{0}}=Z_{1}, \tilde{X_{1}}=X_1\) such that
\(\forall_{i=0,1} \: \tilde{Z_{i}} \odot \tilde{X_{j}} = \delta_{ij}\),
we can use this property to write
\[\forall_{ij \in \{0,1\}^{\otimes 2}} \:\:: \:\: \ket{\beta_{ij}}  \: = \: \tilde{X}_{0}^{i} \tilde{X}_{1}^{j} \:\ket{\beta_{00}}\]
thus obtaining an orthogonal basis for the 2-qubit hilbert space
\(\mathcal{H}^{\otimes 2^{2}}\).

\hypertarget{partially-stabilized-states}{%
\paragraph{Partially Stabilized
States}\label{partially-stabilized-states}}

Up until now we discuss stabilizer states which were fully stabilized in
the sense that we had n independent generators for an n qubit system.
However for most practical circumstances we will be interested in states
which are only partially stabilized i.e will have \(k \: :\:k\leq n\)
independent generators instead. On a \(n\) qubit system with any state
\(\ket{\psi}\) with \(k\) independent stabilizer-generators
\(\mathcal{S} =\{\tilde{Z}_{1}\dots \tilde{Z}_{k}\}\::\: |\mathcal{S}|=k \leq n\)
will be spanned by \(2^{n-k}\) basis vectors of the form
\(\{\ket{\pm \tilde{Z}_{k+1} \dots \pm \tilde{Z}_{n}}\}\).

For instance, consider a 3 qubit system, on which the stabilizer
generators are defined as
\(\mathcal{S} = \{X_{0} Z_{1}, Z_{0}X_{1}Z_{2} \}\), just like before we
can recognize the elements in \(\mathcal{S}\) as
\(\tilde{Z}_{0}= X_{0}Z_{1} \:, \: \tilde{Z}_{1} \:= \: Z_{0}X_{1}Z_{2}\).
We can complete the \(\tilde{Z}\) pauli-operators by picking
\(\tilde{Z}_{2}= Z_{1}X_{2}\) and then pick the \(\tilde{X}\) operators
as \(\forall_{i\in \{0,1,2\}} \:\tilde{X}_{i}= Z_{i}\) such that the
canonical commutation relations
(\(\forall_{i,j} \:\tilde{X}_{i}\odot \tilde{Z}_{j} = \delta_{ij}\)) are
satisfied. Any state \(\ket{\psi}\) that is `partially-stabilized' by
\(\mathcal{S}\) must be spanned by the linear combination of
\(\{\ket{X_{0} Z_{1}, Z_{0}X_{1}Z_{2}, Z_{1}X_{2}} , \ket{X_{0} Z_{1}, Z_{0}X_{1}Z_{2}, -Z_{1}X_{2}} \}\)
i.e
\[\ket{\psi} = \alpha \ket{X_{0} Z_{1}, Z_{0}X_{1}Z_{2}, Z_{1}X_{2}} + \beta \ket{X_{0} Z_{1}, Z_{0}X_{1}Z_{2}, -Z_{1}X_{2}}\]for
arbitrary \(\alpha, \beta\). Realize that even though the system has 3
physical qubits, in essence the state \(\ket{\psi}\) is spanned by just
two basis vectors and thus is an element of a hilbert-space
\(\mathcal{H}^{\otimes 2}\) and thus is equivalent to single qubit. In
the refs. this fact is referred to as \(\ket{\psi}\) encoding just one
`logical qubit'.

For any system of \(k\) stabilizer-generators on an \(n\) dimensional
system, we can have a logical space of \(n-k\) qubits (i.e of dim
\(\mathcal{H}^{2^{n-k}}\)) , correspondingly there will be \(2(n-k)\)
independent pauli-operators that commute with paulis in \(\mathcal{S}\)
we will indicate as \(\mathcal{L}(\braket{\mathcal{S}})\). In the above
example \(\mathcal{L}(\braket{\mathcal{S}})= \{Z_{2}, Z_{1}X_{2}\}\).

Given Stabilizer Group (i.e a commuting set of pauli) we can always
construct a stabilizer hamiltonian of the form
\[H_{\mathcal{S}}= - \sum\limits_{S \in \mathcal{S}} S\]. We can probe
into the eigen spectrum of the hamiltonian by using few simple
properties. Any state \(\ket{\psi}\) we have
\(S \ket{\psi} =\ket{\psi}\) or \(S \ket{\psi}=-\ket{\psi}\) since
\(S \in \mathbb{P}^{\otimes{n}}\) , thus
\[H_{\mathcal{S}}\ket{\psi}=- \sum\limits_{S \in \mathcal{S}}S \ket{\psi} = - \sum\limits_{S\in \mathcal{S}} (-1)^{\psi_{S}}\: \ket{\psi}\]
where \(S \ket{\psi} = -1^{\psi_{S}}\ket{\psi}\) and
\(\forall_{S \in \mathcal{S}} \:: \:\psi_{S}\in \{0,1\}\). If we have
\(k\) generators (i.e \(|\mathcal{S}|=k\)) , then \(H_{\mathcal{S}}\)
has \(\frac{(k+1)!}{k! 1!}=k\) distinct eigenvalues of the form
\(\epsilon(H_{\mathcal{S}}) =\{k, k-2, \dots ,-k\}\). We can enumerate
the eigenvalues simply by choosing the number of stabilizers
\(S \in \mathcal{S}\) that are in their \(+1\) eigenspace i.e we can
partition the stabilizer generators as
\(\mathcal{S} = \mathcal{S^{+}} \cup \mathcal{S^{-}}\) where the set
\(\mathcal{S^+}\) and \(\mathcal{S^{-}}\) correspond to the set of
stabilizers with +1 and -1 eigenspace respectively, the energy
configuration can be represented compactly as
\(\psi_\mathcal{S} = (\psi_{S_{1}}, \psi_{S_{2}}, \dots , \psi_{S_{k}})\:\in \{0,1\}^k\)
. So if \(\epsilon(\psi_{\mathcal{S}}) = E\) we know that
\(|\mathcal{S^{+}|}= \frac{E+k}{2}\) and
\(|\mathcal{S^-}|=\frac{-E+k}{2}\) where there are
\(\frac{k!}{|\mathcal{S}^{+}!||\mathcal{S}^{-}|}\) many configurations
corresponding to the energy level \(E\). Further since an energy
configuration \(\psi_{\mathcal{S}}\) only specifies the state of \(k\)
stabilizer-measurements the corresponding eigenspace has a degeneracy
equivalent to the \(\mathcal{H}^{2^{n-k}}\) dimensional logical space.

From the section
\protect\hyperlink{Actionux5cux2520ofux5cux2520Pauliux5cux2520operatorsux5cux2520onux5cux2520Stabilizerux5cux2520States.md}{\#Action
of Pauli operators on Stabilizer States} we can see that any eigenstate
of \(H_{\mathcal{S}}\) corresponding to the energy configuration
\(\psi_\mathcal{S}\) would transform under the action of pauli operator
\(P\) would transform to another state that has the energy configuration
is
\(\tilde{\psi}_\mathcal{S}= (\forall_{S_{i}\in \mathcal{S}} \: \: (P \odot S_{i})\oplus\psi_{S_{i}})\)
where \(\oplus\) is addition modulo 2 (aka XOR addition).

\textbf{Example :} For Bell states,
\(H_{\beta} = - (X_{1}X_{2} + Z_{1}Z_{2})\) , realize that the state
\(\ket{\beta_{ij}}\) corresponds to the energy configuration
\(\psi_{\beta}= (i,j)\). And has the following eigen-decomposition.
\(\epsilon(\beta_{00})=-2\),
\(\epsilon(\beta_{01}) = \epsilon(\beta_{10}) =0\)
,\(\epsilon(\beta_{11}) = 2\).

\hypertarget{effect-of-perturbation-on-stabilizer-hamiltonians}{%
\subsubsection{Effect of Perturbation on Stabilizer
Hamiltonians}\label{effect-of-perturbation-on-stabilizer-hamiltonians}}

Now that we understand a bit about the structure of the eigenspectrum of
stabilizer hamiltonians we can look into how their eigenspaces are
modified under the influence of external perturbations. Any generic
perturbations represented as the weighted sum of pauli operators as
\(H_{\delta} = \lambda_{1}E_{1} + \lambda_{2}E_{2} + \dots\) where the
weights \(\lambda\) indicate the strength of the perturbations and are
usually of small magnitude. For most physically inspired problems of
interest the error operators \(E_{1}, E_{2} \dots\) are single qubit
paulis, (for instance the problems in
{[}{[}ReadingDocuments/VQE-InfusedCircuitMBQC@LDellantonio24.pdf\textbar VQE-InfusedCircuitMBQC@LDellantonio24{]}{]},
{[}{[}ReadingDocuments/VQE-MBQC@RFegurson21.pdf\textbar VQE-MBQC@RFegurson21{]}{]},
etc.) However we are

\end{document}
