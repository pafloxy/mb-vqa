% Options for packages loaded elsewhere
\PassOptionsToPackage{unicode}{hyperref}
\PassOptionsToPackage{hyphens}{url}
%
\documentclass[
]{article}
\usepackage{lmodern}
\usepackage{amssymb,amsmath}
\usepackage{ifxetex,ifluatex}
\ifnum 0\ifxetex 1\fi\ifluatex 1\fi=0 % if pdftex
  \usepackage[T1]{fontenc}
  \usepackage[utf8]{inputenc}
  \usepackage{textcomp} % provide euro and other symbols
\else % if luatex or xetex
  \usepackage{unicode-math}
  \defaultfontfeatures{Scale=MatchLowercase}
  \defaultfontfeatures[\rmfamily]{Ligatures=TeX,Scale=1}
\fi
% Use upquote if available, for straight quotes in verbatim environments
\IfFileExists{upquote.sty}{\usepackage{upquote}}{}
\IfFileExists{microtype.sty}{% use microtype if available
  \usepackage[]{microtype}
  \UseMicrotypeSet[protrusion]{basicmath} % disable protrusion for tt fonts
}{}
\makeatletter
\@ifundefined{KOMAClassName}{% if non-KOMA class
  \IfFileExists{parskip.sty}{%
    \usepackage{parskip}
  }{% else
    \setlength{\parindent}{0pt}
    \setlength{\parskip}{6pt plus 2pt minus 1pt}}
}{% if KOMA class
  \KOMAoptions{parskip=half}}
\makeatother
\usepackage{xcolor}
\IfFileExists{xurl.sty}{\usepackage{xurl}}{} % add URL line breaks if available
\IfFileExists{bookmark.sty}{\usepackage{bookmark}}{\usepackage{hyperref}}
\hypersetup{
  hidelinks,
  pdfcreator={LaTeX via pandoc}}
\urlstyle{same} % disable monospaced font for URLs
\setlength{\emergencystretch}{3em} % prevent overfull lines
\providecommand{\tightlist}{%
  \setlength{\itemsep}{0pt}\setlength{\parskip}{0pt}}
\setcounter{secnumdepth}{-\maxdimen} % remove section numbering

\author{}
\date{}

\begin{document}

\emph{Here I am summarising some of the rules that we discussed about
extracting pauli-exponentials out of measurement pattern. The man idea
follows from ther analysis presented in {[}RMWsim21{]}
(cf.~\url{LiteratureReview}), but I have eased up the formalism to suit
our specific requirements some of which I disucssed with you.}

\begin{center}\rule{0.5\linewidth}{0.5pt}\end{center}

\hypertarget{algebraic-identities}{%
\subsection{Algebraic Identities}\label{algebraic-identities}}

\hypertarget{projectors}{%
\subsubsection{Projectors}\label{projectors}}

Usually projectors are linear-operators that map a state to a certain
subspace. In our work we are usually interested in projectors that map
to eigenspaces of multi-qubit pauli operators. Given a \(n\)-qubit pauli
\(A \in \mathbb{P}^{\otimes n}\) we can create projector
\(\hat{P}_{aA} = \frac{I+(-1)^{a}A}{2}\) which maps any state to the
\((-1)^a\) eigenspace of the pauli operators \(A\).

\hypertarget{properties}{%
\paragraph{Properties}\label{properties}}

\begin{enumerate}
\def\labelenumi{\arabic{enumi}.}
\tightlist
\item
  \(aA \:\hat{P}_{aA} = \hat{P}_{aA}\) and
  \(\hat{P}_{aA} \:aA = \hat{P}_{aA}\)
\item
  For \(B \in \mathbb{P}^{\otimes n}\),
  \(\hat{P}_{aA} \: B \: = \: B \: \hat{P}_{(a \oplus A \odot B)A}\)
  where \(\oplus\) is addition modulo 2 and \(\odot\) indicates the
  commutation of the two pauli operators (cf \href{Main\%20Note}{Main
  Note}).
\end{enumerate}

\hypertarget{pauli-exponential-sequence}{%
\subsubsection{Pauli Exponential
Sequence}\label{pauli-exponential-sequence}}

For any pauli \(A \in \mathbb{P}\) we can construct a pauli-exponential
\(A(\theta) \: = \: e^{-i \theta A}\). Given any ordered sequence of
paulis \(\mathcal{E}= (A_{1}, A_{2}, A_{3}, \dots A_{t})\) we can
construct an unitary paramterized by the anlge of rotation for each
individual pauli as
\[\mathcal{E}(\theta)  \: = \: A_{1}(\theta_1) A_{1}(\theta_2) \dots A_{1}(\theta_t) \]
Any arbitrary unitary can be expressed as a parameterized exponential
sequence, the most trivial example being the case of arbitrary single
qubit unitary expressed as a sequence of \(X-Z-X\) roation parameterized
by relevant angles.

\hypertarget{properties-1}{%
\paragraph{Properties}\label{properties-1}}

\begin{enumerate}
\def\labelenumi{\arabic{enumi}.}
\tightlist
\item
  The order is not strict, as if there are two commuting paulis then
  their corresponding pauli-exponentiations will also commutes. To
  account for this we instead define a partial order \(\prec\) over the
  paulis, which
  \((\mathcal{E},\prec) = (A_1 \prec A_2, A_{3}\prec \dots \prec A_{t-2}, A_{t-1}, A_t)\)
  captures the fact the elements that are not comparable under the
  \(\prec\) must be commuting.
\item
  For any pauli \(B\), \(A(\theta) \:B = B A( (-1)^{A \odot B}\theta)\)
\item
  \textbf{Product Rotation Lemma :} For any projector \(\hat{P}_{bB}\)
  and pauli-exponential \(A(\theta)\) the following holds
  \[A(\theta) \: \hat{P}_{bB} = AB(b \theta) \:\hat{P}_{bB}\]
\end{enumerate}

\hypertarget{pauli-exponential-extraction}{%
\subsection{Pauli Exponential
Extraction}\label{pauli-exponential-extraction}}

(cf.~{[}RMWsim21{]}) Following our discussion we can realise that any
measurement pattern in the conventional sense can be written as
\[ \prod_{i \in (\mathcal{M}, \prec)} \left(\hat{P}_{m_iM_{i}}E_i(\theta_{i})\right) \: \ket{\mathcal{S}_{G},\psi_{in}}\]
where

\begin{itemize}
\tightlist
\item
  \(\mathcal{S}_{G}\) corresponds to the stabilizers of the graph-state
  and \(\psi_{in}\) is the arbitrary input state.\\
\item
  \(\hat{P}_{m_iM_{i}}\) corresponds the measurement on the \(i\) th
  qubit, \(m_i\) denotes the measurement outcome
\item
  \(E_i(\theta_{i})\) correponds to the rotation prior to the
  measurement,( in {[}RMWsim21{]} this corresponds to eq 5).
\item
  Note that here \(M_i \odot E_i =1\) as otherwise we can move the
  pauli-exponential through the projector
\end{itemize}

The main aim is to extract all the exponentials sandwiched between the
projectors, so that we can find an equivalent unitary operation on the
system. The existence of (focussed)gflow guarantees that this should
always be possible, independent of the choice of the paramaterization
angles \(\theta\).

\textbf{NB:} We will assume that we are using a focussed-gflow in our
discussions, any gflow \((C, \prec)\) if exists in an open-graph can
always be focussed to construct a focussed-gflow \((C_{f}, \prec)\) ,
and as i mentioned focusing a gflow only changes the correction-set and
not the partial order.

If gflow exists we can always find a correction-operator
\(\forall_i \: C_{i}\in \braket{\mathcal{S}_G}\) such that

\begin{enumerate}
\def\labelenumi{\arabic{enumi}.}
\tightlist
\item
  \(C_{i} \odot E_{i}=0\)
\item
  \(C_{i} \odot M_{j}=\delta_{ij}\)
\item
  \(\forall_{k \prec i} \: C_{i}\odot E_{k}=0 \:\implies C_{i}\odot M_{k}=0, \as M_{k}\odot E_k=1\)
\end{enumerate}

Note that since both \(E_{i},M_{i}\) are single qubit operators, thus if
any pauli operator commutes with it then it must have the same operator
or \(I\) on the \(i\)th qubit. The above conditions restrict the
structure of the correction operator to the form
\[C_{i}  \: = \: \left( \tilde{E}_i \right)_{\mathcal{O}}  \otimes \left(M_{j}\otimes M_{k}\otimes .. \right)_{i \prec \mathcal{O}^{c}} \otimes \left( E_{i}  \right)_{i} \],
where

\begin{itemize}
\tightlist
\item
  \(\tilde{E}_i\) can be interpreted as the extracted part of the
  exponential as it acts exclusively on the output qubits
  \(\mathcal{O}\).
\item
  \(\left(M_{j}\otimes M_{k}\otimes .. \right)_{i \prec \mathcal{O}^{c}}\)
  corresponds to requirement 3 the main takeaway from the structure is
  the fact that upon the action of the measurement via \(\hat{P}_{mM}\)
  only the part acting on the output qubits remain and thus allows us to
  construct the unitary that was implemented by the measurement pattern.
\end{itemize}

To see the role of the correction operator consider the following
example
\href{shared-repos/mb-vqa/notes/attatchments/pauli-extraction-martina_annotated.svg}{/notes/attatchments/pauli-extraction-martina_annotated.svg}

The extracted sequence pauli exponentials are corresponds to the
`extracted' unitary and is the overall effect of the measurement process
on the state \(\ket{\mathcal{S}_G, \psi_{in}}\)

\begin{center}\rule{0.5\linewidth}{0.5pt}\end{center}

\hypertarget{todo}{%
\paragraph{Todo}\label{todo}}

\begin{enumerate}
\def\labelenumi{\arabic{enumi}.}
\tightlist
\item
  Try the same procedure for the examples that you come across in
  different papers

  \begin{itemize}
  \tightlist
  \item
    start by finding gflow i.e the correction-sets and the partial order
    \((C, \prec)\)
  \item
    see that the choice of the correction set satisfies the above
    conditions
  \item
    extract the unitary :)
  \end{itemize}
\end{enumerate}

\end{document}
